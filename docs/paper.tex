\documentclass[12pt, twoside]{article}
\usepackage{jmlda}
\usepackage{url}
\usepackage{graphicx}
\newcommand{\hdir}{.}

\begin{document}

\title
    {Восстановление функциональных групп головного мозга с помощью графовых диффузных моделей}
\author
    {А.\,И.~Астахов, С.\,К.~Панченко, В.\,В.~Стрижов} 
\email
    {astakhov.am@phystech.edu; panchenko.sk@phystech.edu; strijov@phystech.edu}
\abstract{
В данной работе рассматривается задача классификации многомерного временного ряда, представляющего собой электроэнцефалограмму головного мозга человека. Стандартные подходы, использующие двухмерные свертки, не могут учесть пространственную структуру сигнала, поскольку датчики, считывающие показатели, находятся на сферической поверхности. В качестве решения предлагается использовать графовое представление функциональных групп, а для моделирования использовать нейронную диффузию.

\keywords{Головной мозг, ЭЭГ, Графовые нейронные сети, диффузионные модели}

\bigskip

\noindent


}

\maketitle
\par

\section{Введение}

Человеческий мозг — сложная система, функциональность которой определяется взаимодействием различных областей. Понимание этих связей важно для нейронаук, диагностики заболеваний и интерфейсов "мозг-компьютер". Современные методы, такие как электроэнцефалография (ЭЭГ), позволяют регистрировать многомерные временные ряды, но их анализ осложняется нерегулярной пространственной структурой и сложными функциональными связями.

Традиционные сверточные нейронные сети (CNN) неэффективны для таких данных, так как рассчитаны на регулярные сетки. Это стимулирует применение графовых нейронных сетей (GNN), которые могут моделировать функциональную связность мозга в виде графа. Узлы графа соответствуют областям мозга, а ребра — силе связей между ними. Однако многие GNN-методы игнорируют временную динамику активности мозга.

В данном исследовании предлагается использование Диффузионной сверточной рекуррентной нейронной сети (DCRNN), которая сочетает графовое моделирование пространственных зависимостей с рекуррентными нейронными сетями (RNN) для анализа временных данных. DCRNN позволяет моделировать как пространственную диффузию сигналов на графе, так и их динамику во времени. Этот подход решает проблемы нерегулярного расположения электродов и неевклидовой природы поверхности мозга.

Экспериментальная проверка проводится на открытых наборах данных. Основные этапы включают предобработку данных, построение графа, обучение модели и оценку её эффективности по метрикам, таким как точность классификации и интерпретируемость функциональных связей. Ожидается, что использование DCRNN позволит выявить сложные пространственно-временные паттерны активности мозга и углубить понимание функциональной связности.


\newpage

\section{Общее описание проблемы}
Дана выборка \( D = (\mathbf{X}, \mathbf{Z}, \mathbf{y}) \), где \(\mathbf{X}\) — набор сигналов, \(\mathbf{Z}\) — координаты электродов, \(\mathbf{y}\) — целевая переменная. Цель — построить модель для декодирования активности мозга на основе иффузионной сверточной рекуррентной нейронной сети (DCRNN).

\subsection{Постановка задачи}

\begin{enumerate}
    \item \textbf{Множество данных}:  
    \[
    \mathbf{X} = [\mathbf{X}_m]_{m=1}^M, \quad \mathbf{X}_m = [\mathbf{x}_t]_{t \in T}, \quad \mathbf{x}_t \in \mathbb{R}^E.
    \]

    \item \textbf{Графовая структура}:  
    \[
    \mathcal{G}(m, t) = \left( \mathcal{V}(m, t), \mathcal{E}(m, t), \mathbf{A}_{\mathbf{X}, \mathbf{Z}}(m, t) \right),
    \]
    где \(\mathbf{A}_{\mathbf{X}, \mathbf{Z}}(m, t)\) — матрица смежности, определяющая веса ребер.

    \item \textbf{Модель DCRNN}:  
    \[
    h_\theta : (\mathbf{X}, \mathbf{A}_{\mathbf{X}, \mathbf{Z}}) \to y, \quad \theta \in \Theta,
    \]
    где \(h_\theta\) — диффузионная сверточная рекуррентная нейронная сеть, учитывающая пространственные и временные зависимости.

    \item \textbf{Функция ошибки}:  
    \[
    \mathcal{L} = -\frac{1}{M} \sum_{m=1}^M \left[ \sum_{c=1}^C \mathbb{I}(y_m = c) \log(p_m^c) \right],
    \]
    где \(p_m^c = h_\theta \left( \mathbf{X}_m, \mathbf{A}_{\mathbf{X}, \mathbf{Z}}(m) \right)\) — вероятность класса \(c\).

    \item \textbf{Оптимизационная постановка}:  
    \[
    \theta^* = \arg\min_{\theta} \mathcal{L}.
    \]
\end{enumerate}

\subsection{Основные термины и определения}
\begin{itemize}
    \item \textbf{Модель}: \( h_\theta \) — диффузионная сверточная рекуррентная нейронная сеть (DCRNN).
    \item \textbf{Решение}: \( \theta^* \) — оптимальные параметры модели.
    \item \textbf{Алгоритм}: Процедура минимизации функции ошибки с использованием градиентного спуска.
\end{itemize}



\bibliographystyle{plain}
\bibliography{Biblio}

\end{document}
